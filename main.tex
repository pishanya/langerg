\documentclass[11pt]{article}

\usepackage{graphicx}
\usepackage{amsfonts}
\usepackage{amsmath}
\usepackage{amsthm}
\usepackage{latexsym}
\usepackage{longtable}
\usepackage{docmute}

\usepackage[T1, T2A]{fontenc}
\usepackage[english,russian]{babel}


\font\rm=larm1095
\font\bf=labx1095



\usepackage{microtype} 

\usepackage{misccorr}
\usepackage[a4paper]{geometry}
\usepackage{tikz}
\usepackage{subfigure}
\usepackage{algpseudocode}
\usepackage[section]{algorithm}
\usepackage{pgfplots}


\geometry{
    layoutheight = 241mm,
    layoutwidth = 171mm,
    layouthoffset = 19.5mm,
    layoutvoffset = 28mm,
    inner = 22mm,
    outer = 22mm,
    top = 18mm,
    bottom = 21mm,
    showcrop
}
\pagenumbering{gobble}
\newtheorem{theorem}{Теорема}
\newtheorem{sled}{Следствие}
\newtheorem{lemma}{Лемма}
\usetikzlibrary{positioning, arrows.meta}
\pgfplotsset{compat = 1.9}
\usepackage{proof}
\usepackage[nottoc]{tocbibind}
\usepackage{amsbib}

\usepackage[utf8]{inputenc}
\usepackage[T2A]{fontenc}

\title{Моделирование взаимодействия в островке Лангерганса с помощью автомата Мура}
\date{}
\begin{document}
\selectlanguage{russian}
\maketitle

\selectlanguage{russian}
\renewcommand{\abstractname}{Аннотация}
\begin{abstract}
  \begin{quotation}
\small
В статье разработана формальная дискретная модель эндокринного островка Лангерганса на основе автомата Мура, в которой входной сигнал представляет собой квантованный уровень глюкозы, а состояния соответствуют конфигурациям активности $\alpha$-, $\beta$- и $\delta$-клеток. Показано, что предложенный подход позволяет однозначно описать ключевые режимы секреции инсулина, глюкагона и соматостатина.

\smallskip
\textbf{Ключевые слова:} автомат Мура, островки Лангерганса, дискретное моделирование.
\end{quotation}

\end{abstract}
\newpage

\tableofcontents
\newpage

\section{Ввведение}

В настоящее время разрабатываются технологии, позволяющие формализовано моделировать метаболические и пластические механизмы человека. Однако многомасштабные модели, охватывающие различные уровни организации живого организма и эволюционирующие на протяжённости нескольких месяцев, обладают высокой сложностью и требуют продолжительных процедур верификации. В частности, любые такие модели должны воспроизводить реакцию системы на внешние раздражители так же адекватно, как это делает здоровый человек в возрасте 25–30 лет.
Для повышения управляемости процесса и ускорения верификации мы предлагаем поэтапный подход: сначала строить частные модели отдельных функциональных субсистем, а затем интегрировать их в единую многоуровневую систему. В рамках настоящей работы сосредоточено внимание на моделировании островков Лангерганса — ключевого звена эндокринной регуляции глюкозного обмена — с использованием формализма конечных автоматов \cite{kudr1}. 
В то же время применение подобных моделей непосредственно в клинической практике затруднено из-за значительных требований к вычислительным ресурсам, объёму оперативной памяти и объёму необходимых баз данных. Поэтому окончательная система должна быть реализована в виде суррогатной модели, компромиссно приближающей поведение биологической системы при ограниченных данных и ресурсах, с возможностью адаптации под индивидуальные особенности пациентов.
Верификация такой комплексной модели также представляет собой серьёзную задачу: требуются как большие объёмы экспериментальных данных, так и персонализированные параметры, выходящие за рамки «средних» показателей популяции. Предлагаемый подход позволяет разделить задачу на более мелкие — верифицируемые компоненты, объединённые в конечный автомат. Изложенное в данной статье решение начинается с построения математической модели основного регулятора энергетического и пластического обмена островка Лангерганса и служит основой для дальнейшей интеграции и масштабирования на более крупные временные масштабы.

Однако следует сделать оговорку: всякая модель отдельного органа или отдельной части метаболизма, которую в дальнейшем нужно будет использовать в общей системе других моделей, которые описывают метаболизм, должна допускать синхронизацию (соответствие временных масштабов) с ними. 
В связи с этим при моделировании островка Лангерганса следует предусмотреть моделирование этого объекта в нескольких временных масштабах, исходя из различных требований к использованию этих моделей. Изменение временных масштабов моделирования будет требовать одновременно и требовать, и предоставлять возможность более менее подробного структурного моделирования островка.

Поясним подробнее последнюю мысль. Островок Лангерганса представляет собой объемное трехмерное клеточное образование, состоящее из десятков клеток и отделенное от основное массы тканей поджелудочной железы, иннервированное и содержащее часть кровеносной капиллярной системы. Клетки островка Лангерганса относятся к различным типам, выполняют различные функции и являются источниками различных экскретируемых метаболитов (гормонов).

Поэтому иерархия разрабатываемых моделей на наиболее полном уровне, описывающем функционирование островка в полном объеме, должна завершаться построением трехмерной многоклеточной модели. Математическим аппаратом для такого описания могут служить инструменты вычислительной геометрии, теории графов и дифференциальные уравнения различных типов. Однако в данной статье мы фокусируемся на изучении внешнего функционала островка Лангерганса как регулятора уровней инсулина и глюкагона. При этом нашей целью является понимание механизма этого автоматизма, что также требует учета секреции соматостатина, выполняющего важную модулирующую роль внутриостровковых взаимодействий \cite{brereton15}. Модель, описывающая все аспекты функционирования островка, смогла бы обеспечивать наиболее высокие требования к фундаментальному исследованию объекта, но потребовала бы значительных вычислительных ресурсов. Такая детализированная модель вряд ли может быть использована в системах, описывающих метаболизм человека на организменном уровне одновременно с динамическими моделями других органов человеческого тела, таких как мышцы, печень, жировая ткань и т.д. Причиной этому, помимо высокой сложности этих совокупностей моделей, будут также разные временные масштабы характерных процессов: характерные константы времени переключения управляющих воздействий в островке Лангерганса значительно меньше, чем характерные временные константы процессов метаболизма в прочих органах.


\subsection{Островок Лангерганса}

Островки Лангерганса представляют собой небольшие округлые скопления эндокринных клеток \cite{zhang23, gustafsson08, otter16}, разбросанные по всей паренхиме поджелудочной железы, среди экзокринной ткани \cite{gustafsson08, otter16}. Хотя они составляют всего около 1–2\% объёма поджелудочной железы (некоторые источники указывают 2-3\% \cite{gustafsson08}), на их долю приходится от 10 до 15\% её кровотока \cite{gustafsson08}. Каждый островок имеет диаметр от десятков до нескольких сотен микрометров (например, 50-250 µm \cite{gustafsson08}) и содержит от нескольких десятков до нескольких тысяч клеток \cite{gustafsson08, otter16}. Островки окружены капиллярами и тонкой соединительной тканью, отделяющей их от экзокринных ацинусов \cite{brereton15, gustafsson08}. Такой дискретный характер организации делает островки микроорганами, специализированными для секреции гормонов непосредственно в кровь \cite{gustafsson08, guelph}.
\subsection{Анатомия островков Лангерганса}
Каждый островок имеет характерную для вида пространственную организацию \cite{brereton15}. У мышей и крыс типична ядро-мантия архитектура: инсулин-секретирующие $\beta$-клетки занимают центральную часть, а глюкагон-секретирующие $\alpha$-клетки и соматостатин-секретирующие $\delta$-клетки формируют оболочку по периферии \cite{brereton15}. В отличие от этого, в человеческих островках отсутствует строгая сегрегация клеток: $\alpha$-, $\beta$- и $\delta$-клетки рассеяны по всему островку вперемешку \cite{brereton15, gustafsson08}. Такая мозаичная 3D-организация обеспечивает, во-первых, непосредственное прилегание всех эндокринных клеток к капиллярам \cite{brereton15} и, во-вторых, плотные контакты между разными типами клеток \cite{brereton15, zhang23}. В результате каждая эндокринная клетка человека получает равный доступ к кровоснабжению и может эффективно передавать паракринные сигналы соседним клеткам и получать их \cite{brereton15, zhang23}. Считается, что правильная архитектоника островка способствует синхронной секреции гормонов и эффективной регуляции гомеостаза глюкозы \cite{brereton15, guelph}. Нарушение нормальной структуры островков отмечается при всех типах сахарного диабета и сопровождается дисфункцией гормональной секреции \cite{brereton15}.

Островки Лангерганса обладают чрезвычайно густой сетью капилляров \cite{gustafsson08, otter16} – плотность капиллярного русла в них в 5–10 раз выше, чем в окружающей экзокринной ткани поджелудочной железы. Каждый островок получает кровь через одну или несколько приносящих артериол, которые входят в его центр в составе нейроваскулярного пучка \cite{gustafsson08}. Внутри островка артериолы разветвляются на капилляры, оплетающие каждую эндокринную клетку, после чего кровь собирается в выносящие венулы \cite{brereton15, gustafsson08}. Такой тип кровоснабжения обеспечивает быстрый выход гормонов в кровоток: инсулин и другие гормоны секретируются непосредственно в просвет капилляров \cite{guelph}. Несмотря на малый вклад в массу поджелудочной, островки получают непропорционально большой объём крови (не менее 10\% панкреатического кровотока \cite{gustafsson08}), что подчёркивает их важную эндокринную функцию. Богатая васкуляризация также создаёт микросреду с высоким парциальным давлением кислорода и питательных веществ, необходимую для чувствительности клеток к изменениям гликемии. Кроме того, капилляры островка выстланы фенестрированным эндотелием, облегчающим быструю диффузию глюкозы в клетки и гормонов в кровь \cite{gustafsson08, otter16}.

Поджелудочная железа и её островки снабжены автономными нервными волокнами, включающими симпатические (адренергические) и парасимпатические (холинергические) аксоны \cite{brereton15, gustafsson08}. Нервные волокна обычно входят в островок вместе с кровеносными сосудами (в составе нейроваскулярного пучка) и ветвятся внутри него \cite{gustafsson08}. Иннервация островков человека хорошо подтверждена морфологически: нервные окончания обнаруживаются в контакте с $\beta$-клетками и другими эндокринными клетками \cite{brereton15, gustafsson08}. Парасимпатические (вагусные) импульсы стимулируют в островках секрецию инсулина и, одновременно, высвобождение глюкагона, что характерно для фазы подготовки организма к приёму пищи \cite{gustafsson08, guelph}. Симпатическая стимуляция через норадреналин, напротив, подавляет высвобождение инсулина и усиливает секрецию глюкагона – это важно в условиях гипогликемии или стресса, когда требуется мобилизация глюкозы \cite{gustafsson08, guelph}. Интересно, что в островках человека $\alpha$-клетки способны выделять ацетилхолин, который действует как местный нейромедиатор и стимулирует секрецию инсулина $\beta$-клетками и соматостатина $\delta$-клетками \cite{brereton15}. Таким образом, нейронные и паракринные сигнальные механизмы тесно переплетены, образуя комплексную систему регуляции внутриостровкового гомеостаза \cite{brereton15, zhang23, guelph}.

\subsubsection{Эндокринные клетки островков}
Панкреатические островки содержат пять основных типов эндокринных клеток, каждая из которых синтезирует свой гормон и выполняет уникальную роль в регуляции метаболизма \cite{gustafsson08, otter16}. Количественное соотношение этих клеток в островках человека примерно следующее: $\beta$-клетки (бета-клетки) составляют около 60–70\% (разные источники дают от 50\% до 75\% \cite{brereton15, gustafsson08, guelph}) всех эндокринных клеток, $\alpha$-клетки (альфа-клетки) – около 20–30\% (например, 20-26\% \cite{gustafsson08, guelph}), $\delta$-клетки (дельта-клетки) – 5–10\% (например, 4-8\% \cite{gustafsson08, guelph}), PP-клетки (также называемые $\gamma$- или F-клетки) – до 5\% (0.3-1\% \cite{gustafsson08, guelph}), а $\epsilon$-клетки (эпсилон-клетки) – менее 1\%. В рамках данной работы мы ограничимся рассмотрением $\alpha$-, $\beta$- и $\delta$-клеток. Выбор $\alpha$- и $\beta$-клеток обусловлен их ключевой ролью в гомеостазе глюкозы через секрецию глюкагона и инсулина соответственно \cite{brereton15, guelph}. $\delta$-клетки включены в анализ, поскольку они выполняют важную модулирующую функцию внутри островка, секретируя соматостатин, который паракринно регулирует активность как $\alpha$-, так и $\beta$-клеток, предотвращая их чрезмерную секреторную активность и поддерживая баланс в системе \cite{brereton15}. Прочие гормоны и секретируемые вещества островков Лангерганса выходят за рамки данной публикации. Ниже приведена характеристика каждого из рассматриваемых типов клеток, включая их локализацию, гормоны и взаимодействия.
\subsubsection{$\alpha$-клетки}
Альфа-клетки составляют примерно четверть всех клеток островка (20-26\% \cite{gustafsson08, guelph}) и продуцируют гормон глюкагон, являющийся главным контринсулярным гормоном \cite{brereton15, guelph}. У грызунов $\alpha$-клетки преимущественно локализованы по периферии островка, однако в человеческих островках они рассеяны диффузно, располагаясь по соседству с $\beta$-клетками в толще островка \cite{brereton15}. Основной стимул для секреции глюкагона – снижение уровня глюкозы в крови (голодание, физическая нагрузка) \cite{brereton15, guelph}, а также симпатическая стимуляция \cite{guelph}. Глюкагон выполняет сигнализирующую роль «тревоги» при гипогликемии: он вызывает быстрый подъем уровня глюкозы за счёт стимуляции гликогенолиза и глюконеогенеза в печени \cite{brereton15, guelph}. В экспериментах показано, что экзогенный глюкагон способен значительно повышать концентрацию глюкозы в крови при введении натощак. Кроме того, глюкагон оказывает паракринное влияние внутри островка. С одной стороны, он может стимулировать $\beta$-клетки через повышение внутриклеточного cAMP и тем самым усиливать секрецию инсулина при достаточном уровне глюкозы (этот эффект выражен в постпрандиальном состоянии) \cite{brereton15, guelph}. С другой стороны, при низком содержании глюкозы глюкагон способствует мобилизации резервов, а не секреции инсулина. Таким образом, $\alpha$- и $\beta$-клетки образуют динамическую пару: глюкагон в определённых условиях стимулирует выброс инсулина, который затем по принципу отрицательной обратной связи подавляет дальнейшую секрецию глюкагона, предотвращая избыточный подъем гликемии \cite{brereton15, guelph}. Помимо этого, на $\alpha$-клетки воздействуют $\delta$- и $\beta$-клетки: инсулин (и амилин) $\beta$-клеток, а также соматостатин $\delta$-клеток угнетают секрецию глюкагона, обеспечивая его снижение при гипергликемии \cite{brereton15, guelph}. В норме эти механизмы поддерживают взаимно обратную регуляцию инсулина и глюкагона: при повышении глюкозы усиливается инсулиновая секреция и тормозится выпуск глюкагона, тогда как при падении глюкозы – наоборот, глюкагон выходит в кровь, а инсулин угнетается \cite{guelph}.


\subsubsection{$\beta$-клетки}
Бета-клетки являются наиболее многочисленной и функционально главной популяцией островка, составляя более половины его клеток \cite{brereton15, gustafsson08, guelph}. Они обычно образуют скопления в центре островка (особенно у грызунов), однако у человека $\beta$-клетки также широко перемешаны с другими типами клеток и встречаются по всему объёму островка \cite{brereton15, gustafsson08}. Основной гормон $\beta$-клеток – инсулин, ключевой анаболический гормон, снижающий уровень глюкозы в крови \cite{gustafsson08, guelph}. Инсулин секретируется в ответ на гипергликемию, поступление глюкозы и аминокислот, а также под действием инкретинов и парасимпатической стимуляции \cite{gustafsson08, otter16, guelph}. Выделяясь в кровь, инсулин способствует поглощению глюкозы мышечной и жировой тканью, стимулирует синтез гликогена в печени и ингибирует продукцию глюкозы печенью, тем самым предотвращая гипергликемию после приёма пищи \cite{guelph}. Кроме инсулина, $\beta$-клетки секретируют со-гормон амилин (IAPP) \cite{gustafsson08}, который участвует в регуляции скорости всасывания глюкозы (замедляет опорожнение желудка) и угнетает секрецию глюкагона после еды. Амилин также действует паракринно, дополняя тормозящее влияние инсулина на $\alpha$- и $\delta$-клетки. Внутри островка $\beta$-клетки выполняют важную сигнальную роль: инсулин служит мощным локальным ингибитором $\alpha$-клеток – показано, что нейтрализация инсулина или гибель $\beta$-клеток приводит к резкому возрастанию секреции глюкагона \cite{brereton15, guelph}. Таким образом, высокая концентрация инсулина в окрестностях $\beta$-клеток при повышенной гликемии напрямую подавляет продукцию глюкагона, предотвращая одновременное действие контринсулярных сигналов. С другой стороны, $\beta$-клетки улавливают сигналы от $\alpha$- и $\delta$-клеток: глюкагон $\alpha$-клеток через рецепторы GLP-1/глюкагона на $\beta$-клетках способен усиливать глюкозо-стимулированную секрецию инсулина \cite{brereton15, zhang23, guelph}, а соматостатин $\delta$-клеток ограничивает чрезмерный выброс инсулина, выполняя роль локального «тормоза» \cite{brereton15, guelph}. Морфологически $\beta$-клетки приспособлены к своему секреторному функционированию: они обычно поляризованы вблизи капилляров, где располагаются гранулы с инсулином, обеспечивая быструю экзоцитозную отдачу гормона в кровь \cite{brereton15, gustafsson08}.

\subsubsection{$\delta$-клетки}
Дельта-клетки составляют относительно небольшой процент клеток островка (около 4-8\% у человека \cite{gustafsson08, guelph}) и распределены по всему островку, часто на границах скоплений $\alpha$- и $\beta$-клеток \cite{brereton15}. Они секретируют гормон соматостатин (SST) – мощный универсальный ингибитор секреции многих желез \cite{brereton15, gustafsson08, guelph}. Соматостатин, выделяемый $\delta$-клетками, действует паракринно на соседние клетки островка, подавляя высвобождение инсулина $\beta$-клетками, глюкагона $\alpha$-клетками, а также панкреатического полипептида PP-клетками \cite{brereton15, guelph}. Кроме того, соматостатин тормозит собственную секрецию $\delta$-клеток через аутокринный отрицательный механизм обратной связи \cite{brereton15}. Благодаря стратегическому положению и длинным отросткам, дельта-клетки могут контактировать с несколькими $\beta$- и $\alpha$-клетками одновременно, осуществляя координирующую роль \cite{brereton15}. Секреция соматостатина стимулируется повышением уровня глюкозы (схоже с инсулином) и поступлением питательных веществ, то есть активируется в постпрандиальном состоянии \cite{brereton15}. Возникает вопрос: зачем нужен соматостатин, если при гипергликемии требуется усиленная секреция инсулина? Считается, что соматостатин выполняет функцию «буферного гормона», предотвращая чрезмерное выделение как инсулина, так и глюкагона и поддерживая баланс в системе \cite{brereton15}. Например, при резком повышении глюкозы соматостатин освобождается одновременно с инсулином, быстро подавляя остаточную секрецию глюкагона (что преимущественно важно на высоких уровнях гликемии) и умеренно ограничивая дальнейший выброс инсулина, не допуская его избыточного избытка \cite{brereton15}. В условиях же низкой глюкозы уровень соматостатина падает, снимая тормоз с $\alpha$- и $\beta$-клеток. Таким образом, $\delta$-клетки служат модератором внутриостровковых взаимодействий, обеспечивая стабильность метаболической реакции на изменяющиеся условия \cite{brereton15}. Следует также отметить, что помимо своего локального (внутриостровкового) паракринного действия, соматостатин, поступая в кровоток, оказывает и системное (эндокринное) влияние на различные органы и ткани организма. Например, он ингибирует секрецию гормона роста гипофизом, а также подавляет секреторную активность и моторику желудочно-кишечного тракта, тем самым участвуя в более широкой координации метаболических процессов и пищеварения в организме \cite{brereton15}.

\subsubsection{Доставка инсулина и глюкагона}
Кровь, оттекающая от поджелудочной железы, собирается в панкреатических венулах, которые впадают в селезёночную и верхнюю брыжеечную вены, образуя систему воротной вены печени \cite{gustafsson08}. Таким образом, гормоны островков Лангерганса – прежде всего инсулин и глюкагон – поступают не прямо в большой круг кровообращения, а сначала попадают в печень через воротную вену \cite{gustafsson08}. Эта портальная доставка обеспечивает уникальный механизм приоритетного воздействия гормонов на печень – главный орган, регулирующий баланс глюкозы и других энергетических депо организма \cite{brereton15, gustafsson08, guelph}. Концентрации инсулина и глюкагона в воротной вене значительно выше, чем в периферической крови: у здоровых людей уровень инсулина в воротном русле превышает периферический минимум на 50\%, а уровень глюкагона – на 30–40\% \cite{walter80}. Такое градиентное распределение связано с эффектом первого прохождения через печень: около половины инсулина (некоторые данные указывают до 80\% \cite{gustafsson08}), секретируемого $\beta$-клетками, извлекается (метаболизируется) печенью при первом же прохождении крови. В результате печень выступает в роли «фильтра» и первичного адресата для инсулина, защищая периферические ткани от чрезмерной гипогликемической реакции, но обеспечивая при этом высокую локальную концентрацию инсулина для стимуляции поглощения глюкозы и синтеза гликогена в гепатоцитах. Глюкагон, секретируемый $\alpha$-клетками, аналогично поступает по портальной системе прямо в печень, где реализует свои действия – стимулирует распад гликогена и глюконеогенез \cite{brereton15, guelph}. За счёт портальной циркуляции печень постоянно экспонируется высокому соотношению инсулина к глюкагону, особенно после еды, что способствует накоплению гликогена и синтезу жира в постпрандиальный период. В периферическом кровообращении же отношение инсулина к глюкагону ниже, что предотвращает чрезмерное потребление глюкозы мышцами в ущерб мозгу. Таким образом, воротная вена играет ключевую физиологическую роль в распределении эндокринных сигналов поджелудочной железы: она направляет инсулин и глюкагон сперва в печень, тем самым координируя обмен глюкозы на уровне всего организма \cite{gustafsson08}. Нарушения этого механизма (например, при заболеваниях печени или при экзогенном введении инсулина подкожно, минуя портальный путь) могут приводить к несоответствующему портально-периферическому градиенту гормонов и, как следствие, к нарушению метаболической регуляции.

\subsection{Постановка задачи}
(оговорка: вставить текст Саши и описать его модель)
Постановка задачи заключается в разработке формальной дискретной модели эндокринного островка Лангерганса, способной в конечном итоге служить низкочастотным переключателем в динамических моделях метаболизма организма.

Промежуточная (высокочастотная) модель формируется на базе конечного автомата Мура, детально описывающего быстрые переключения между режимами активности $\alpha$-, $\beta$- и $\delta$-клеток в ответ на изменение уровня глюкозы. Каждое состояние автомата отражает конфигурацию секреции инсулина, глюкагона и соматостатина, а переходы задаются пороговыми уровнями гликемии. Эта модель обеспечивает ясность структуры сигналов и простоту алгоритмической реализации без использования систем обыкновенных дифференциальных уравнений.


Редукция до низкочастотной модели. По результатам анализа динамики промежуточной модели проводится редукция дискретного автомата с сохранением его основных внешних функций и частотных характеристик, соответствующих диапазонам, в которых модель будет применяться в системах моделирования метаболических процессов. Итоговая сверхпростая модель должна функционировать как управляющий переключатель («ключ») в низкочастотных органных и системных динамических моделях.

Целью исследования является получение и верификация дискретного конечного автомата, обладающего частотными свойствами, совместимыми с динамикой метаболических моделей организма.

\section{Методы}
В данной работе для моделирования одной и той же системы — островка Лангерганса — используются два взаимодополняющих подхода. Во-первых, применяется дискретный метод на основе теории конечных автоматов: управляемый модуль реализован как автомат Мура, что позволяет формально задать набор дискретных режимов и делать выходной сигнал функцией исключительно текущего состояния. Во-вторых, строится непрерывная модель в виде системы обыкновенных дифференциальных уравнений (ОДУ) ...(оговорка: дописать Саша и ЕА).

\subsection{Применение обыкновенных дифференциальных уравнений для описания регуляторной динамики островка Лангерганса}
Примечание: нужно сослаться на работу, в которой происходит моделирование с помощью ОДУ. Затем идет описание метода Александра.

Предполагаем, что каждого типа клеток есть два состояния: они включены и вырабатывают определенный гормон, либо выключены, тогда гормон начинает тратиться. 
Эти режимы моделируются системой дифференциальных уравнений, решением которых является функция [\ref{eq:main_solution}].

\begin{equation}\label{eq:main_solution}
f(t) = \begin{cases}
  1-e^{-t}, &\text{если клетка работает}\\
   e^{-t}, &\text{если клетка не работает}
 \end{cases}
\end{equation}

Включение и выключение клетки происходит по логике персептона: на вход подаются сигналы, от которых зависит активация клетки [\ref{fig:all_schema}] (здесь ссылка на картинку модели). Амплитуда сигналов активации (непрерывная линия) идут с положительным знаком, а сигналов деактивации (пунктирная линия) -- с отрицательным.

Если суммарное значение входных сигналов больше порогового, то клетка активируется, иначе блокируется. Таким поведением моделируется химическое сопротивление, которое не пропускает сигнал в нейроне до достижения пороговой амплитуды.   

\begin{figure}[H]
  \centering
  \includegraphics[width=0.8\textwidth]{all_schema.png}
  \caption{Модель $\alpha$ клетки в островке Лангерганса.}
  \label{fig:all_schema}
\end{figure}

\subsection{Конечные автоматы}
Необходимо перейти к регуляторному описанию, как управляющего автомата. Конечные автоматы позволяют строго описывать системы, чьё поведение можно представить в виде набора конечных состояний и переходов между ними. Такой подход широко применяется в цифровой электронике, теории программирования и моделировании биологических процессов, где важно явно задать возможные режимы работы и условия их переключения. Ниже приведено формальное определение детерминированного конечного автомата, которое будет служить основой для дальнейшего конструирования дискретной части модели. Более подробно про конечные автоматы изложено в \cite{kudr1}.

\emph{Детерминированным конечным автоматом} называется упорядоченная пятёрка
\[
  \mathcal{A} = \bigl(Q,\,\Sigma,\,\delta,\,q_0,\,F\bigr),
\]
где
\begin{itemize}
  \item $Q=\{q_1,\dots,q_n\}$ — конечное множество состояний;
  \item $\Sigma=\{a_1,\dots,a_m\}$ — конечный входной алфавит;
  \item $\delta\colon Q\times\Sigma\to Q$ — функция переходов;
  \item $q_0\in Q$ — начальное состояние;
  \item $F\subseteq Q$ — множество допускающих (терминальных) состояний.
\end{itemize}
Работа автомата происходит в дискретные моменты времени $t=0,1,2,\dots$. При поступлении
во вход символа $a(t)\in\Sigma$ в момент $t$ автомат, находясь в состоянии $q(t)$, переходит
в состояние
\[
  q(t+1)=\delta\bigl(q(t),\,a(t)\bigr).
\]
Автомат $\mathcal{A}$ \emph{допускает} слово $a(0)a(1)\dots a(k)$ тогда и только тогда,
когда $q(k+1)\in F$.

В модели эндокринного островка Лангерганса выход однозначно определяется конфигурацией активности клеток, то есть состоянием системы. Автоматы Мура позволяют формализовать каждый устойчивый режим секреции, комбинацию активности $\alpha$-, $\beta$- и $\delta$-клеток, как отдельное состояние с фиксированным выходом. Это обеспечивает большую наглядность, упрощает верификацию и минимизацию модели. Поэтому в дальнейшем мы сосредоточимся именно на автоматах Мура.

\subsection{Автомат Мура}

\emph{Автомат Мура} задаётся шестёркой
\[
  M = \bigl(Q,\,\Sigma,\,\Gamma,\,\delta,\,\lambda,\,q_0\bigr),
\]
где
\begin{itemize}
  \item $Q=\{q_1,\dots,q_n\}$ — конечное множество состояний;
  \item $\Sigma=\{a_1,\dots,a_m\}$ — входной алфавит;
  \item $\Gamma=\{b_1,\dots,b_p\}$ — выходной алфавит;
  \item $\delta\colon Q\times\Sigma\to Q$ — функция переходов;
  \item $\lambda\colon Q\to\Gamma$ — выходная функция;
  \item $q_0\in Q$ — начальное состояние.
\end{itemize}
Автомат Мура работает в дискретные моменты времени $t=0,1,2,\dots$. В момент~$t$ на вход
подаётся символ $a(t)\in\Sigma$, после чего выполняется переход
\[
  q(t+1)=\delta\bigl(q(t),\,a(t)\bigr),
\]
и одновременно на выходе формируется значение
\[
  b(t)=\lambda\bigl(q(t)\bigr).
\]
Фиксируя $q(0)=q_0$ и входное слово $a(0)a(1)\dots a(k)$, однозначно получаем
последовательности состояний $q(1),\dots,q(k+1)$ и выходных символов
$b(0),\dots,b(k)$.

Для наглядного представления функционирования автомата Мура используют ориентированный граф $G=(V,E)$, в котором каждая вершина соответствует состоянию $q_i\in Q$, а каждое ребро $e=(q_i,q_k)\in E$ соединяет текущее состояние $q_i$ с новым состоянием $q_k=\delta(q_i,a_j)$ по входному символу $a_j\in\Sigma$. Таким образом, метка каждого ребра - это входной сигнал $a_j$.


\section{Моделирование островка Лангерганса}
\subsection{Реализация дифференциальной модели}

Модель всего островка Лангерганса на дифференциальных уравнениях была реализована с помощью программы Matlab Simulink. 

Схема модели:

\begin{figure}[H]
  \centering
  \includegraphics[width=0.8\textwidth]{alpha_ceil_schema.png}
  \caption{Модель $\alpha$-клетки в островке Лангерганса. DGB - decreased glucose bound }
  \label{fig:alpha_ceil_schema}
\end{figure}

Как можно заметить, $\alpha$-клетка активируется, когда суммарное воздействие пониженного уровня глюкозы  "переборит" влияние бета и гамма клеток и достигнет порогового значения химического сопротивления. 

\subsection{Реализация конечного автомата}
В настоящей работе для дискретизации уровня глюкозы вводится переменная
$$
G \in \{-1,0,+1\}.
$$
Её значения трактуются так:
$$
G = 
\left\{
  \begin{array}{ll}
    -1, & \text{пониженный уровень глюкозы},\\
     0, & \text{нормальный уровень},\\
    +1, & \text{повышенный уровень}.
  \end{array}
\right.
$$
При численном моделировании непрерывная функция уровня глюкозы \( \tilde G(t)\), например, синусоидального вида, затем квантуется по этому правилу, что даёт дискретную траекторию \(G(t)\). Такое кодирование упрощает функцию переходов: каждая дискретная метка $G$ запускает свои правила переключения клеток.


Рассмотрим автомат Мура
\[
  M = \bigl(Q,\,A,\,B,\,\delta,\,\lambda,\,q_0\bigr),
\]
где
\[
  Q = \{0,1\}^3
    = \bigl\{(s_\alpha,s_\beta,s_\delta)\mid s_i\in\{0,1\}\bigr\}
\]
— множество внутренних состояний (включённость или выключенность
$\alpha$-, $\beta$- и $\delta$-клеток Лангерганса);

\[
  A = \{-1,0,+1\}
\]
— входной алфавит, соответствующий дискретному уровню глюкозы $G$;

\[
  B = \{0,1\}^3
\]
— выходной алфавит, совпадающий с вектором состояния клеток;

\[
  \delta\colon Q\times A\;\longrightarrow\;Q
\]
— функция переходов, определяющая обновление состояния по входному уровню
глюкозы $G$;

\[
  \lambda\colon Q\;\longrightarrow\;B,\qquad
  \lambda(q)=q
\]
— функция выходов (выдаёт текущее состояние $q$);

\[
  q_0\in Q
\]
— начальное состояние.

В качестве примера рассмотрим следующий такт работы автомата:
\[
  q(t) = (1,1,0), \qquad G(t) = +1.
\]
Тогда по определению функций переходов и выходов имеем
\[
  \begin{aligned}
    q(t+1) &= \delta\bigl(q(t),G(t)\bigr)
            = \delta\bigl((1,1,0),+1\bigr)
            = (1,1,1), \\
    b(t)   &= \lambda\bigl(q(t)\bigr)
            = \lambda\bigl((1,1,0)\bigr)
            = (1,1,0), \\
    b(t+1) &= \lambda\bigl(q(t+1)\bigr)
            = \lambda\bigl((1,1,1)\bigr)
            = (1,1,1).
  \end{aligned}
\]



В соответствие с дискретизацией уровня глюкозы, а также с учетом взаимодействия $\alpha$, $\beta$ и $\delta$ клеток был построен конечный автомат, моделирующий работу островка Лангерганса. Построенный автомат изображен на рис.\ref{fig:graph}.

\begin{figure}[htbp]
  \centering
  \includegraphics[width=1.2\textwidth]{graph.png}
  \caption{Конечно-автоматная модель Мура взаимодействия $\alpha$, $\beta$ и $\delta$ клеток в островке Лангерганса.}
  \label{fig:graph}
\end{figure}

\subsection{Реализация редуцированного автомата}

При попытке описать островок Лангерганса на одном уровне детализации сразу «микросекундных» процессов (закрытие $K_{ATP}$-каналов, вливание $Ca^{2+}$, экзоцитоз) и «минутных» и «часовых» трендов (смена доминанты глюкагона и инсулина) мы сталкивались с необходимостью выставлять экстремально малый шаг интегрирования. Это приводило к росту вычислительной нагрузки и трудностям с устойчивостью численного решения. В то же время слишком подробная модель «тонет» в биохимических флуктуациях, и её трудно связать с основными режимами работы островка.

Вместо полного перебора восьми комбинаций активности трёх типов клеток внимание было сконцентрировано только на двух физиологически значимых популяциях — $\alpha$- и $\beta$-клетках, а также было выделелено три состояния автомата:
\begin{itemize} \item \textbf{«$\alpha$-режим»}: активна преимущественно глюкагоновая секреция; \item \textbf{«$\beta$-режим»}: доминирует инсулиновая реакция; \item \textbf{«Совмещенный режим»}: одновременная активность обеих популяций. \end{itemize}

На рис.~\ref{fig:graph_simple} показана упрощённая диаграмма состояний и переходов автомата, позволяющая одновременно моделировать быстрые внутриклеточные события и медленные макро-тренды без избыточных вычислений и потери биологической интерпретации.


\begin{figure}[htbp]
  \centering
  \includegraphics[width=1.0\textwidth]{graph_simple.png}
  \caption{Редуцированный конечный автомат взаимодействия $\alpha$, $\beta$ и $\delta$ клеток.}
  \label{fig:graph_simple}
\end{figure}


Результат работы автомата изображен на рис.\ref{fig:simulation_simple}

\begin{figure}[htbp]
  \centering
  \includegraphics[width=1.0\textwidth]{simulation_simple.png}
  \caption{Результат моделирования упрощенного автомата взаимодействия $\alpha$, $\beta$ и $\delta$ клеток.}
  \label{fig:simulation_simple}
\end{figure}

\section{Результаты}

\subsection{Модель на диффурах}

Так как данная модель имеет интегратор, то она выдает результат работы клеток, а именно выработку гормонов глюкагона, инсулина и соматостатина.

\begin{figure}[htbp]
  \centering
  \includegraphics[width=1.0\textwidth]{sin_model_res.jpg}
  \caption{Результат моделирования интегральной схемы взаимодействия $\alpha$, $\beta$ и $\delta$ клеток.}
  \label{fig:sin_model_res}
\end{figure}

\subsection{Модель на автоматах}
Результаты моделирования детализированного автомата в режиме тактовой работы при синусоидально изменяющемся уровне глюкозы представлены на рис. \ref{fig:simulation}. Отображённые выходные траектории демонстрируют чёткие прямоугольные импульсы, соответствующие поочерёдной активации и деактивации $\alpha$‑, $\beta$‑ и $\delta$‑клеток. Такая булева логика переключений полностью соответствует механизму взаимодействия клеток, который был описан в биологическом введении, и позволяет детально анализировать внутриклеточные механизмы паракринной регуляции.

Вместе с тем первоначальная задача предполагала разработку дискретного управляющего модуля для координации работы непрерывной модели метаболизма, функционирующей на существенно более медленных временных интервалах (секунды–часы).

Во избежание этих ограничений была выполнена редукция структуры автомата: построена феноменологическая модель, отражающая только внешнюю функциональность островка — зависимость уровней гормонов от текущей гликемии. Для устранения негладких переходов и обеспечения плавности сигналов введён интегратор первого порядка, сглаживающий выходные импульсы. Итоговая упрощённая модель сохраняет ключевые динамические характеристики островка Лангерганса, обеспечивает устойчивую работу при различных сценариях изменения глюкозы и эффективно взаимодействует с ОДУ‑модулем метаболизма.

\subsection{Результат интегратора}

Результатом работы конечнo-автоматной модели островка Лангерганса являются промежутки времени, когда та или иная клетка включена. Если проинтегрировать данные промежутки с учетом расхода вещества на нужды организма, то получим результат на рис. \ref{fig:integrator}
 
\begin{figure}[htbp]
  \centering
  \includegraphics[width=1.0\textwidth]{integrator.png}
  \caption{Интегральный результат работы автомата взаимодействия $\alpha$, $\beta$ и $\delta$ клеток.}
  \label{fig:integrator}
\end{figure}

\begin{figure}[htbp]
  \centering
  \includegraphics[width=1.0\textwidth]{simulation.png}
  \caption{Результат моделирования автомата взаимодействия $\alpha$, $\beta$ и $\delta$ клеток.}
  \label{fig:simulation}
\end{figure}

\section{Заключение}


В работе предложена дискретная модель эндокринного островка Лангерганса на основе автомата Мура. Основной результат заключается в том, что с помощью формализма конечных автоматов удалось чётко выделить и воспроизвести три ключевых режима секреции гормонов: доминирование $\beta$‑клеток, доминирование $\alpha$‑клеток и арбитраж $\delta$‑клеток. Верификация полной и редуцированной версии автомата показала, что редуцированная модель обеспечивает значительное снижение вычислительной сложности без потери адекватности описания переходов между режимами, сохраняя основные биологические закономерности.


Предложенный дискретный подход демонстрирует высокую эффективность и точность моделирования режимов секреции в островках Лангерганса и может стать универсальным блоком в сложных мультикомпонентных моделях углеводного обмена.





\begin{thebibliography}{10}

\bibitem{kudr1}
Кудрявцев~В.Б., Алешин~С.В., Подколзин~А.С. {\it Введение в теорию автоматов}. Москва: Наука, 1985. — 320~с.

\bibitem{walter80}
Walter~R.M.~Jr., Gold~E.M., Michas~C.A., Ensinck~J.W. Portal and peripheral vein concentrations of insulin and glucagon after arginine infusion in morbidly obese subjects~// {\it Metabolism}. 1980. Vol.~29, No.~11. P.~1037–1040. DOI:~10.1016/0026-0495(80)90213-9.

\bibitem{brereton15}  
Brereton~M.F., Vergari~E., Zhang~Q., Clark~A.  
Alpha-, Delta- and PP-cells: Are They the Architectural Cornerstones of Islet Structure and Co-ordination?~//  
{\it Journal of Histochemistry \& Cytochemistry}. 2015. Vol.~63, No.~8. P.~575–591.

\bibitem{zhang23}
Zhang~Q., Huising~M.O., Da~Silva~Xavier~G., Hauge-Evans~A.C. Editorial: The pancreatic islet a multifaceted hub of inter-cellular communication~// {\it Frontiers in Endocrinology}. 2023. Vol.~14. P.~1182897.

\bibitem{gustafsson08}
Gustafsson~A.J., Islam~M.S. Islets of Langerhans: cellular structure and physiology~// In N.~Ahsan (Ed.), {\it Chronic Allograft Failure: Natural History, Pathogenesis, Diagnosis and Management}. Landes Bioscience, 2008.

\bibitem{otter16}  
Otter~S., Lammert~E.  
Exciting Times for Pancreatic Islets: Glutamate Signaling in Endocrine Cells~//  
{\it Trends in Endocrinology \& Metabolism}. 2016. Vol.~27, No.~3. P.~167–177.

\bibitem{guelph}
Human Physiology – Glucose Regulation. {\it University of Guelph. Pressbooks}. (n.d.)

\end{thebibliography}




\end{document}
